
\documentclass[french]{article}
\usepackage[utf8]{inputenc}
\usepackage[T1]{fontenc}
\usepackage{babel}
\usepackage{blindtext}
\usepackage[margin=1in]{geometry}

\usepackage[useregional]{datetime2}

\title{CS-330 Projet 2020}   % type title between braces
\author{Christophe MARCIOT, Titouan RENARD}         % type author(s) between braces


\begin{document}
\maketitle

\section{Introduction}

\subsection{Goal}

What is intended in this project is to create a tool capable of predicting wheter a patient suffers from a cardiovascular disease given a set of informations.
The informations given for each patients are :
	\begin{itemize}
		\item A variable called \emph{target}, that takes value $0$ or $1$ indicating if the patient does actually suffer from a cardiovascular diesaes or not (here we use the mediacal convention. That is that 0 means negative and 1 means positive in the detection of cardiovascular disease). This is the variable whose behavior we try to predict.
		\item 13 other variables called respectively \emph{sex}, \emph{cp}, \emph{trestbps}, \emph{chol}, \emph{fbs}, \emph{restecg}, \emph{tahlach}, \emph {exang}, \emph{oldpeak}, \emph{slope}, \emph{ca} and \emph{thal} that can vary between $0$ and $4$ - with minor exceptions.
	\end{itemize}
From now on, we will refer to those $14$  variables as \emph{attributes} and the number taken by those attributes as \emph{values} of the given attribute for the patient. In the end, the tool should also be able to justify its diagnosis, to advise treatements when a disease is diagnosed and be able to treat with continuous data.

\subsection{Structure of the project}
The project is fragmented in $5$ main tasks :
	\begin{itemize} 
		\item Task $1$ : Create a tree capable of guessing the value of \emph{target} for a given patient using a first set of data - given in the file \emph{train\_bin.csv} - using the \emph{ID3} algorithm,
		\item Task $2$ : Test the accuracy of the tree obtained in Task $1$ on a second set of data - given in the file \emph{test\_public\_bin.csv},
		\item Task $3 $: Deduce rules from the tree produced in Task $1$. Design a function capable of predicting the value of \emph{target} using only the aformentionned rules and use them to give a justification of the prediction based on the rules used,
		\item Task $4$ : Improve Task $3$. Design a tool that is capable of the same tasks as in $3$, but it must also be able to give a treatement for the patient when the person is diagnosed with a disease and justify said treatment,
		\item Task $5$ : Improve the \emph{ID3} implementation of Task $1$ such that the new implementation is capable of dealing with continuous data instead of only discrete ones. The construction of the tree uses the data provided in \emph{train\_continuous.csv} and the test of the accuracy uses the data provided in \emph{test\_public\_continuous.csv}.
	\end{itemize}

\section{Discussion of the results}

\subsection{Task $1$}

\subsection{Task $2$}

\subsection{Task $3$}

\subsection{Task $4$}

\subsection{Task $5$}

\section{Conclusion}

\end{document}